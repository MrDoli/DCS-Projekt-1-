\documentclass[12pt, a4paper]{article}

\usepackage[T1]{fontenc}
\usepackage[utf8]{inputenc}
\usepackage[english, polish]{babel}
\usepackage{polski}
\usepackage{graphicx}
\usepackage[export]{adjustbox}
\usepackage{amsmath}
\usepackage{pdfpages}
\usepackage{enumerate}

\begin{titlepage}
\author{
	\\ Maria Konieczko
	\\ Alicja Poturała
	\\ Marcin Dolicher
	\\
	\\ AIR Semestr V
}
\title{
    \quad \quad DCS i SCADA
	\newline
	Sprawozdanie z Projektu I		
}
\date{}
\end{titlepage}

\begin{document}
\maketitle
\newpage
\tableofcontents

\newpage
\section{Zagadnienia i założenia projektowe}
Do wykonania były dwa zadania obowiązkowe, które polegały na identyfikacji modeli statycznych i dynamicznych. W tym celu używaliśmy metody najmnnieszjych kwadratów. Dane zostały przygotowane w trzech plikach danestat8.zip, danedynucz8.zip i danedynwer8.zip. W przypadku danych statycznych sami musieliśmy podzielić dane na zbiory uczący i weryfiujący, a dla danych dynamicznych zbiory te były już przygotowane.

\section{Identyfikacja obiektu}

\section{Struktura}

\section{Strojenie PID}

\subsection{f) }
\begin{figure}[h!]
%\includegraphics[scale=0.55, center]{}
\end{figure}

\section{Testowanie}

\section{Wnioski}





















































\end{document}

