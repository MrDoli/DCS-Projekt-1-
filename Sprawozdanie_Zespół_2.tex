\documentclass[12pt, a4paper]{article}

\usepackage[T1]{fontenc}
\usepackage[utf8]{inputenc}
\usepackage[english, polish]{babel}
\usepackage{polski}
\usepackage{graphicx}
\usepackage[export]{adjustbox}
\usepackage{amsmath}
\usepackage{pdfpages}
\usepackage{enumerate}
\usepackage{siunitx}


\begin{titlepage}
\author{
	\\ Maria Konieczko
	\\ Alicja Poturała
	\\ Marcin Dolicher
	\\
	\\ AIR Semestr V
}
\title{
    \quad \quad DCS i SCADA
	\newline
	Sprawozdanie z Projektu I		
}
\date{}
\end{titlepage}

\begin{document}
\maketitle
\newpage
\tableofcontents

\newpage
\section{Zagadnienia i założenia projektowe}
Postawione przed nami zadanie polegało na zaprojektowaniu regulatora PID, który steruje obiektem grzewczym, w naszym przypadku będzie to grzałka. Na obiekt działają zakłócenia w postaci wiatru generowanego przez wiatrak. Punkt pracy jest ustawiony na $30 \%$ mocy grzałki co daje nam stałą temperaturę w okolicach \ang{38}. Projekt regulatora i testy dla obiektu (zmiana zakłóceń i wartości zadannej) zostały przeprowadzone przy użyciu programów dostarczonych przez firmę Ovation. Do zebrania odpowiedzi skokowej wykorzystane zostało oprogramowanie MATLAB. Charakterystyka obiektu odpowiadała obiektowi 1 na 1 plus 1 (1 wejście, 1 wyjście i 1 zakłócenie). Regulator miał za zadanie utrzymywać zadaną wartość dla grzałki przy zmiennych wartościach zakłóceń. Ocena jakości regulacji polega na obliczaniu błędu średniokwadratowego dla sygnału sterującego w porównaniu do wartości zadanej. Najlepszy regulator został wyłoniony na podstawie konkursu na ostatnich zajęciach. 
\section{Identyfikacja obiektu}

\section{Struktura}

\section{Strojenie PID}

\subsection{f) }
\begin{figure}[h!]
%\includegraphics[scale=0.55, center]{sciezka do pliku}
\end{figure}

\section{Testowanie}

\section{Wnioski}





















































\end{document}

